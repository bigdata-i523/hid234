\documentclass[sigconf]{acmart}

\usepackage{graphicx}
\usepackage{hyperref}
\usepackage{todonotes}

\usepackage{endfloat}
\renewcommand{\efloatseparator}{\mbox{}} % no new page between figures

\usepackage{booktabs} % For formal tables

\settopmatter{printacmref=false} % Removes citation information below abstract
\renewcommand\footnotetextcopyrightpermission[1]{} % removes footnote with conference information in first column
\pagestyle{plain} % removes running headers

\newcommand{\TODO}[1]{\todo[inline]{#1}}

\begin{document}
\title{Big Data Analytics in Tourism Industry}


\author{Weixuan Wang}
\orcid{1234-5678-9012}
\affiliation{%
  \institution{Indiana University Bloomington}
  \city{Bloomington} 
  \state{Indiana} 
  \postcode{47405}
}
\email{wangweix@indiana.edu}


% The default list of authors is too long for headers}
\renewcommand{\shortauthors}{W. Wang}


\begin{abstract}
This study focused on how the tourism industry has been impacted by the
development of the Internet and improvements in information and communication
technologies. This study explored how big data are generated related the tourism
industry and how big data analytic has influenced and can further affect tourism
research.
\end{abstract}

\keywords{I523, HID234, Big data analytic, Tourism, }


\maketitle

\section{Introduction}

Information Communication Technologies (ICTs) have been transforming tourism business
globally and revolutionizing the world of Tourism. It transforms tourism from a
labor-intensive to an information-intensive industry\cite{Williams201787}.
Developments in search engines, capacity, and speed of networks have influenced the
number of tourists around the world that use technologies for planning and
experiencing their travels. In addition, ICTs enable travelers to access reliable and
accurate information and make reservations faster, cheaper and more convenient than
the traditional way \cite{chung2009}.


The development of ICTs also enables Internet users to both create and distribute
content (multimedia information), which is called user-generated content (UGC) or
consumer-generated content (CGC)\cite{chung2009}.Platforms for UGC or CGC such as
blogs, virtual communities, wikis, social networks, collaborative tagging, and media
sharing sites play an increasingly important role as information sources for tourists.
Social networks like TripAdvisor, Instagram, Facebook, Yelp, and booking.com are
essential for tourists for multiple reasons: for their preparation of trips (booking
hotels), during a trip (choose restaurant) and sharing their experience (writing hotel
or restaurant reviews). Millions of data records are produced daily in regards to
tourism by tourists, businesses an public services \cite{Shafiee16}. These data can be
distinguished from big data by its volume, velocity (the speed it is produced),
variety (different formats), variability (diversity of sources) and volatility
(different level of production) \cite{MIAH2017}. 

Big data has been attracting more and more attention from tourism business and tourism
researchers alike \cite{GUO2017467}. Big data analytics which is the activities of the
specification, capture, storage, access, and analysis of such data sets to make sense
of its content, provide new opportunities and challenges for tourism practitioners and
researchers to understand tourists' behavior \cite{MIAH2017}. This study explores how
big data are generated in the tourism industry and used in tourism research, further
explore the implication and influence of big data and big data analytics for future
research. 

\section{Big Data in the Tourism Industry}
Most activists in tourism industry had been generating a huge amount of data for
several years.booking a plane ticket, reserving a hotel room and renting a car all
leaves a data trail \cite{Shafiee16}. These data could add up to more than hundred of
terabytes or petabytes structured data in the conventional databases. Discussion of
travel arrangement on travel community online, status and post on social media like
facebook and twitter, compliments and compliant on review websites like TripAdvisor
constructs as more challenging and live unstructured data that arrives at a much
faster pace than a conventional database \cite{akerkar2012}. Tourism practitioners are
trying to understand tourists' behavior by accepting and analyzing these big data
\cite{Shafiee16}.

Airline and hotel chains have been using their big data which is the large volume of
structured information that has been produced internally. Airlines and hotels have
employed this tool to analyze of hotels' prices. Moreover, airlines have optimized the
details of planning for the crew and routing \cite{Shafiee16, GJT14}.  
The online sector of the tourism industry has also quickly adopted big data to improve
internal decisions and understand customers \cite{akerkar2012}. The online sector of
the industry include online travel agencies (like Kayak, Expedia), meta-search engines
(like Google) and some information companies that distribute tourism information
(TripAdvisor).For example, Amadeus has developed a program for special results and the
ability to search for its customers and Kayak has developed a program for predicting
the prices \cite{Shafiee16}.

\subsection{Use of Social Media data in the Tourism Industry}
Tourists in the digital age often use a variety of tools to access information that
the tourism industry or other users have provided \cite{XIANG2015120}. A tourist
produces high volume of data when they are searching for travel websites, reporting
data on mobile applications, sharing traffic information in the cities, searching and
posting on social media, taking and sharing photos, reporting experience on travel
websites and social media \cite{akerkar2012, Shafiee16}. All these data that are
produced constantly can demonstrate tourists' motivation, interests, and their
planning patterns and so on.

Previous studies have demonstrated several different usage and formats of big data in
the travel and tourism industry \cite{XIE2017101}. Social media is one of them that
has a huge effect on the tourism industry. Social media includes blogs, review sites,
media sharing, social networks, and wikis. The remarkable growth of these data sources
has inspired new strategies to understand the socio-economic phenomenon in various
fields \cite{Shafiee16}. Discussions on social media are considered as electronic
word-of-mouth (eWOM) that has in some degree substituted tradition face-to-face
word-of-mouth for information exchange of tourist experience \cite{chung2009}.
According to a study on travelers' counseling with social media for travel planning in
the US in 2014, 44 percent in the age group 18-34 years old use information in social
media before planning for travel \cite{statistica17a}.


Photo post on photographic sharing website also can also provide extensive information
on the tourists. Previous studies have connected photos posted on Panoramio, Flickr,
and Instagram \cite{GJT14, MIAH2017}. Because when a tourist post pictures on these
websites, their photo is tagged with geographic locations and ordered chronologically.
Therefore analyzing photos posted by tourists can provide a photo density map to
better understand tourists' behaviors, and potentially provide opportunities to detect
atypical tourists behavior and characterize communities behaviors. However, the study
also has its own limitation because of the limitation of technology to better exploit
the data \cite{GJT14}. Another study focused on the sequence of locations in shared
geotagged photos by tourist to identify and recommend travel routes which helped the
travel recommender system to generate personalized recommendation according to
interests and time available \cite{kurashima2013travel}. 

\subsection{Other Big Data in the Tourism Industry}
Beside the use of social media content to analyze tourists behavior, previous studies
by Statistics Netherlands has also proposed using other innovative ways to understand
tourists behavior by using mobile phone \cite{heerschap2014innovation}. First method
is using log data collected by an app installed on mobile devices, which allowed
researchers to tract accurate movements of a person or family. This app also can pop
up different questions that be triggered by location or change of time, such as
purpose of the journey, satisfaction and activities. This innovative design combined
the traditional survey with log data from smartphone measurements produced a rich and
valuable sets of data \cite{heerschap2014innovation}. However, this kind of method may
be hard to get willing participants, because of privacy concerns and also technical
issues such as people may not know how to download and use such application.

Another project from Statistic Netherlands uses aggregated mobile phone meta-data
based on call detailed records from 2012 to 2014. This study collaborated with two
telecom providers. Call detailed records contained information of the date and time
and location where a communication through mobile network is used. The study uses
these information and roaming data to identify unique foreign tourists, was able to
detect different groups of foreign tourists and what are their favorite touristic
sites within Netherlands \cite{heerschap2014innovation}.The limitation of this
research is also restricted because it requires collaboration with telecom providers
and its privacy concerns. With the technology development and widespread of WI-FI,
when tourists go to another country they may not need to have roaming service in their
destination \cite{heerschap2014innovation}.   

\section{Big Data in Tourism Research}
Although tourism scholar has recognized the importance of UGC data such as travel
blogs, online reviews and social media post as a form of eWOM has a huge influence in
creating destination image. Tourism scholar has also done content analysis on online
reviews and travel blogs, but recognizing big data and using big data in tourism
research is still limited \cite{Williams201787,chung2009}. 

Most tourism research utilizing big data are still focusing on CGC or UGC, especially
online reviews for a hotel. A recent study conducted by Guo, Barnes and Jia used data
mining approach and linguistic analysis to extract meaning from 266,544 online reviews
for 25,670 hotels. They mined their customer review data from TripAdvisor using a web
crawler. Through their linguistic analysis of their data and cross-comparing with
perceptual mapping of the hotels, they find 19 controllable dimensions that are
important for hotels to manage their interactions with visitors (such as the price for
value, check in and check out) \cite{GUO2017467}.

Another study also focused on UGC and trying to find out determinants of hotel
customer satisfaction by discriminating among customers by language group. This study
collected 412,784 reviews on TripAdvisor for 10,149 hotels in China. They have found
out that tourists speaking different languages (such as Chinese, English, German,
French, Russian etc.) differs substantially in terms of their emphasis on various
attributes of hotels, and forming different satisfaction rating for hotels
\cite{LIU2017554}.


Both of the two studies mentioned above were from tourism or hospitality journals,
were conducted by tourism researchers. Another study from outside of tourism research
cohort provided a different study using big data to understand tourist behavior. This
study designed and evaluated a big data analytics method using geotagged photos shared
by tourists on Flickr to support destination management organization in analyzing and
predict tourist behavior patterns at destinations (for this study itis Melbourne,
Australia). The study designed a geotagged photo analytic artifact with textual
meta-data processing geographical data clustering, representative photo identification
and time series data modeling. This study demonstrated how to analyze unstructured big
data to enhance strategic decision making in tourism destinations, provided insight on
how city tour can be designed to better reflect tourists' interests and enrich their
travel experience \cite{MIAH2017}. 


\section{Conclusion}
This study has explored the literature of big data and its implication in the tourism
industry. Both tourism practitioners and tourism researcher has recognized the
influence of big data and big data sources for tourism development. Big data in the
tourism industry are generated by tourists directly, compared to traditional data sets
that are gathered from surveys. Therefore, big data presented us opportunities to
better understand tourist behavior, their motivations, and interests. However, big
data also poses challenges for tourism practitioner and tourism researchers. 

Like these two studies from tourism and hospitality journals, they share similarities
in terms of data collection methods. Tourism researchers have recognized the
importance of user-generated data which was able to provide them the volume of data
they need for better generalization. One limitation of this kind of tourism research
is that they only focus on hotel reviews, but their method could extend to other
tourism sectors such as attraction and event to evaluate or review dimensions of
tourist satisfaction. Another limitation they have is that they are only focusing on
the text-based data from review website. How to integrating and getting useful
information from other unstructured data such as image, video, post on Facebook and
Twitter is still challenging for tourism researchers. However, studies outside of
tourism domains can be helpful in helping tourism researchers to utilize other formats
of big data to understand tourist behavior. Therefore, collaboration with other fields
and utilizing unstructured big data, and big data analytics in relation to tourism are
much needed for tourism research. 


\begin{acks}

  The author would like to thank Dr. Gregor von Laszewski and I523.

\end{acks}

\bibliographystyle{ACM-Reference-Format}
\bibliography{report} 

\section{Bibtex Issues}
\todo[inline]{Warning--empty chapter and pages in editor00}
\todo[inline]{(There was 1 warning)}
\section{Issues}

\DONE{Example of done item: Once you fix an item, change TODO to DONE}




\subsection{Formatting}

    \TODO{Incorrect number of keywords or HID and i523 not included in the keywords}
    \TODO{Other formatting issues}

\subsection{Writing Errors}


    \TODO{Spelling errors}



\end{document}
