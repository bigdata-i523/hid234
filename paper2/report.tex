\documentclass[sigconf]{acmart}

\usepackage{graphicx}
\usepackage{hyperref}
\usepackage{todonotes}

\usepackage{endfloat}
\renewcommand{\efloatseparator}{\mbox{}} % no new page between figures

\usepackage{booktabs} % For formal tables

\settopmatter{printacmref=false} % Removes citation information below abstract
\renewcommand\footnotetextcopyrightpermission[1]{} % removes footnote with conference information in first column
\pagestyle{plain} % removes running headers

\newcommand{\TODO}[1]{\todo[inline]{#1}}

\begin{document}
\title{Big Data and Cloud Computing in Health Informatics for People with Disabilities}


\author{Weixuan Wang}
\orcid{1234-5678-9012}
\affiliation{%
  \institution{Indiana University Bloomington}
  \city{Bloomington} 
  \state{Indiana} 
  \postcode{47405}
}
\email{wangweix@indiana.edu}


% The default list of authors is too long for headers}
\renewcommand{\shortauthors}{W. Wang}


\begin{abstract}
This is my abstract.
\end{abstract}

\keywords{Big Data, Cloud Computing, Disability informatics, Health informatics, i523, HID234 }


\maketitle



\section{Introduction}
People with disabilities are a group that has been overlooked for a long time. Some might question that why
we should care about people with disability. According to UTHealth, as of February 2015, there are about a 
billion people with disabilities in the world and in the United States alone, there are 56.7 million people
with disabilities \cite{Lex}. Notably, the number of people with disabilities is expected to increase as a 
result of extension of human life-span, decreases in communicable diseases, the improvement of medical
technology, and decrease of child mortality \cite{SMITH1987376}. According to United Nation, people with 
disabilities are the largest minority group in the world \cite{Appleyard2005,DARCY2010816,Lex}. While some 
forms of disabilities might be genetic, but temporary or permanent disabilities can happen to anyone, such 
as spinal cord injury after car accident, or limited mobility at later stage of life \cite{Lex}. Population
aging trend also signifies that disability will be a more common and urgent issue in the future \cite{Grue}. 
Therefore, improving the living condition and quality of life for people with disabilities are extremely important to everyone.

There are many different definition of disabilities from different organizations. The most cited official
definition is the 1976 definition of the World Health Organization \cite{Appleyard2005}: ``An impairment is
any loss or abnormality of psychological, physiological or anatomical structure or function; a disability
is any restriction or lack (resulting from an impairment) of ability to perform an activity in the manner
or within the range considered normal for a human being; a handicap is a disadvantage for a given
individual, resulting from an impairment or a disability, that prevents the fulfillment of a role that is
considered normal (depending on age, gender and social and cultural factors) for that individual''. While
people with disabilities are those people who have limitations in their actions or activities resulting
from physical or mental impairments, however, there are many types and levels of disabilities and their
actions and activities are affected differently by their disabilities \cite{Appleyard2005}. The complexity
of disabilities presents difficulties and challenges to accommodate the different needs of people with
disabilities and improve their qualities of life \cite{Datapop}. 


The development of digital technology has changed many people's lives, the life of people with disabilities
has also been improved by technology \cite{Datapop}. People with poor visions can using cell phones to
contact others, access information online with screen readers. People with hearing problems can text other
people with their cell phone. This study is trying to help people with disabilities from a health
informatics prospective, specifically the disability informatics, and looking into how big data and cloud
computing can help monitor and evaluate and understand people with disabilities and help improve their
quality of life. 


\section{Types of Disability}
Disability has different function types and levels of degrees, while these types are not completely exclusive, most of functional types of disability can be categorized into three groups: mobility, sensory, and cognitive \cite{Appleyard2005}. This section provides a simple overview of these three functional types of disabilities and its challenges for people with disabilities. 

Mobility problems are faced by people with physical motor disabilities (such as spinal cord injury after traumatic injury), and people have impaired muscle controls \cite{SMITH1987376}. These people might have problem using technologies than are design to assist them such as wheelchairs or computer interface aids \cite{Appleyard2005}.

Sensory disability includes both visual and aural impairment \cite{Appleyard2005}. Their conditions can range from correctable (such as using eyeglass or hearing aids) to not correctable (such as blind or deaf) \cite{Riga13}. Braille has been traditionally used by people with visual impairment, but now was replaced by technology such as voice synthesis and screen readers \cite{Riga13}.

Cognitive disabilities in general refer to people with cognitive impairment who have difficulties than an average individual with one or more types of mental tasks involving language, memory, perception, problem-solving, hand-eye coordination, conceptualizing, attention and executive functions\cite{Appleyard2005}. 

\section{Disability informatics}
Disability informatics is a sub-specialty of health informatics that is defined as ``any application that
collects, manages, and distributes information that are related to people with disabilities, as well as to
care givers (including familiar members and health care providers) and rehabilitation professionals''
\cite{Appleyard2005}. Disability informatics is closely related to other health informatics areas such as
medical informatics, public health informatics and consumer health informatics, because people with
disabilities usually have some secondary medical condition such as poor health status and increased
personal health care needs. Gather medical and health information can help to better understand and
accommodate people with disabilities \cite{Riga13}. A study from the early 2000 has identified the
potential of public health informatics for prevention at all vulnerable points in the causal chains leading
to disability and proposed that  applications should not be restricted to particular social, behavioral,
or environmental contexts, but in a more global context \cite{Yasnoff}.  Another previous research has
designed and deployed an extended version of Artemis system (a cloud system designed to acquire data and
store physiological data of clinical information for real-time analytics) in a hospital. They have
identified that high speed physiological data produced at intensive care units as big data, and the proper
use of such data can promote health, reduce mortality and disability rates of critical condition patients
and create new cloud-based health analytics \cite{Khazaei14}. Research also has shown that many
disabilities are genetic, therefore, bioinformatics has implications in the education of genetic screening
and gen therapy treatments in the future \cite{Appleyard2005}.

 The information technology applications and issues in disability informatics are categorized into four
 areas: virtual, personal, physical and social/intellectual.
 
\subsection{Virtual Environment}

WWW accessibility (Marriott accessible design) (assistive tool)
 
\subsection{Personal Environment}


Health and Disability Information and Education
Health monitoring

\subsection{Physical Environment}
ADA
smart home \cite{dewsbury03}
urban design

\subsection{Social Environment}
ADA and big data

\section{Cloud Computing and Disability Informatics}

\section{Conclusion}

limitation: there are limited studies on disability informatics and big data. Most of them is illustrating the potential of the area (however these potential studies have been around for decades. The potential and suggestions made by previous studies may result in other product instead of products for people with disabilities(smart home). 


\begin{acks}

  The author would like to thank Dr. Gregor von Laszewski for his
  support and suggestions to write this paper.

\end{acks}

\bibliographystyle{ACM-Reference-Format}
\bibliography{report} 

\appendix


\end{document}
