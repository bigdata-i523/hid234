\documentclass[sigconf]{acmart}

\input{format/i523}

\begin{document}
\title{Big Data and Cloud Computing in Health Informatics for People with Disabilities}


\author{Weixuan Wang}
\orcid{1234-5678-9012}
\affiliation{%
  \institution{Indiana University Bloomington}
  \city{Bloomington} 
  \state{Indiana} 
  \postcode{47405}
}
\email{wangweix@indiana.edu}


% The default list of authors is too long for headers}
\renewcommand{\shortauthors}{W. Wang}


\begin{abstract}
This study provided a short overview of disability informatics, providing examples of using health informatics
for people with disabilities.This study also explored the potentials of using big data sources to better understand 
people with disabilities and uncovered the potential of big data and cloud computing in developing assistive 
technology and information technology for people with disabilities. 
\end{abstract}

\keywords{Big Data, Cloud Computing, Disability informatics, Health informatics, i523, HID234 }


\maketitle



\section{Introduction}
People with disabilities are a group that has been overlooked for a long time. Some might question that why
we should care about people with disability. According to UTHealth, as of February 2015, there are about a 
billion people with disabilities in the world and in the United States alone, there are 56.7 million people
with disabilities \cite{Lex}. Notably, the number of people with disabilities is expected to increase as a 
result of extension of human life-span, decreases in communicable diseases, the improvement of medical
technology, and decrease of child mortality \cite{SMITH1987376}. According to United Nation, people with 
disabilities are the largest minority group in the world \cite{Appleyard2005,DARCY2010816,Lex}. While some 
forms of disabilities might be genetic, but temporary or permanent disabilities can happen to anyone, such 
as spinal cord injury after car accident, or limited mobility at later stage of life \cite{Lex}. Population
aging trend also signifies that disability will be a more common and urgent issue in the future \cite{Grue}. 
Therefore, improving the living condition and quality of life for people with disabilities are extremely
important to everyone.

There are many different definition of disabilities from different organizations. The most cited official
definition is the 1976 definition of the World Health Organization \cite{Appleyard2005}: ``An impairment is
any loss or abnormality of psychological, physiological or anatomical structure or function; a disability
is any restriction or lack (resulting from an impairment) of ability to perform an activity in the manner
or within the range considered normal for a human being; a handicap is a disadvantage for a given
individual, resulting from an impairment or a disability, that prevents the fulfillment of a role that is
considered normal (depending on age, gender and social and cultural factors) for that individual''. While
people with disabilities are those people who have limitations in their actions or activities resulting
from physical or mental impairments, however, there are many types and levels of disabilities and their
actions and activities are affected differently by their disabilities \cite{Appleyard2005}. The complexity
of disabilities presents difficulties and challenges to accommodate the different needs of people with
disabilities and improve their qualities of life \cite{Datapop}. 


The development of digital technology has changed many people's lives, the life of people with disabilities
has also been improved by technology \cite{Datapop}. People with poor visions can using cell phones to
contact others, access information online with screen readers. People with hearing problems can text other
people with their cell phone. This study is trying to help people with disabilities from a health
informatics prospective, specifically the disability informatics, and looking into how big data and cloud
computing can help monitor and evaluate and understand people with disabilities and help improve their
quality of life. 


\section{Types of Disability}
Disability has different function types and levels of degrees, while these types are not completely exclusive, 
most of functional types of disability can be categorized into three groups: mobility, sensory, and cognitive
\cite{Appleyard2005}. This section provides a simple overview of these three functional types of disabilities 
and its challenges for people with disabilities. Mobility problems are faced by people with physical motor 
disabilities (such as spinal cord injury after traumatic injury), and people have impaired muscle controls 
\cite{SMITH1987376}. These people might have problem using technologies than are design to assist them such 
as wheelchairs or computer interface aids \cite{Appleyard2005}.

Sensory disability includes both visual and aural impairment \cite{Appleyard2005}. Their conditions can range
from correctable (such as using eyeglass or hearing aids) to not correctable (such as blind or deaf) \cite{Riga13}. 
Braille has been traditionally used by people with visual impairment, but now was replaced by technology such as 
voice synthesis and screen readers \cite{Riga13}.

Cognitive disabilities in general refer to people with cognitive impairment who have difficulties than an average 
individual with one or more types of mental tasks involving language, memory, perception, problem-solving, 
hand-eye coordination, conceptualizing, attention and executive functions\cite{Appleyard2005}. 

\section{Disability informatics}
Disability informatics is a sub-specialty of health informatics that is defined as ``any application that
collects, manages, and distributes information that are related to people with disabilities, as well as to
care givers (including familiar members and health care providers) and rehabilitation professionals''
\cite{Appleyard2005}. Disability informatics is closely related to other health informatics areas such as
medical informatics, public health informatics and consumer health informatics, because people with
disabilities usually have some secondary medical condition such as poor health status and increased
personal health care needs. Gather medical and health information can help to better understand and
accommodate people with disabilities \cite{Riga13}. A study from the early 2000 has identified the
potential of public health informatics for prevention at all vulnerable points in the causal chains leading
to disability and proposed that  applications should not be restricted to particular social, behavioral,
or environmental contexts, but in a more global context \cite{Yasnoff}.  Another previous research has
designed and deployed an extended version of Artemis system (a cloud system designed to acquire data and
store physiological data of clinical information for real-time analytics) in a hospital. They have
identified that high speed physiological data produced at intensive care units as big data, and the proper
use of such data can promote health, reduce mortality and disability rates of critical condition patients
and create new cloud-based health analytics \cite{Khazaei14}. Research also has shown that many
disabilities are genetic, therefore, bioinformatics has implications in the education of genetic screening
and gen therapy treatments in the future \cite{Appleyard2005}. People with disabilities usually need some 
assistive technology in their daily life. These technology that assist them to perform basic physical and 
social functions. The use information technology and assistive applications in disability informatics are 
categorized into three areas: virtual, personal, physical.
 
\subsection{Virtual Environment}
The digital revolution had and will continue to have a profound positive impact on the life of people with
disability by empowering them with the help of digital technologies \cite{Appleyard2005}. However, there are
still access issues in the digital world. One of the barrier is the use of the World Wide Web (WWW or Web).
The WWW has always had a strong awareness and been advocacy for accessibility since early on in its
evolution. The World Wide Web Consortium (W3C) had passed the Web Accessibility Initiative (WAI) and Web
Content Accessibility Guidelines in the late 1990s \cite{Appleyard2005}. A number of assistive technologies
were designed to help people with disabilities to use the Web. For example BBC Education Text to Speech
Internet Enhancer (BESTIE) is a CGI Perl script that can help people with disabilities who are using
text-to-speech systems for Web browsing to modified the web page removing images, Java and Javascript code
that may cause difficulties to understand the BBC web page content \cite{Erra}. 
However, the limitation of BESTIE is that it is only compatible with BBC website. Other researchers also
came up with Personalizable Accessible Navigation (PAN), which is a set of edge services designed to improve
Web pages accessibility which allow  personalization and the opportunities to select multiple profiles,
making it compatible for web as well as mobile devices \cite{info:doi/10.2196/mhealth.3956}.

\subsection{Personal Environment}
 Disability informatics also emphasis on providing safe personal environment for people with disabilities,
 Health monitoring is a very important area. Technologies like small tracking device can monitor heart rate,
 blood pressure, also allow people to call for help easily and smart clothing and even smart furniture have
 been developed to monitor people's health status and can help provide people with disabilities a safer
 personal environment and also provide health information for their medical care providers
 \cite{Appleyard2005}. However the ethic of such health monitor devices are always in debate, some believe
 it can be an invasion of privacy and a restriction of personal freedom, others hold the ground that its
 main purpose is to help people with disease or disabilities, since it can alert their caregiver if the
 individual are exposed to harm (such as a person with mental disability and has a history of self-harming,
 these device can prevent unwanted behavior) \cite{cunningham2017cloud}. 
 

\subsection{Physical Environment}
Since the American Disabilities Act passed in the 1990s, the accessibility of physical environment has been
improved in a great degree. However, people with disabilities still would meet some barrier and problem, 
one of them is the lack of curb cuts. Assistive information technologies has been developed in an effort to
solve this problem. One of them is MAGUS, which is a project using geographical information system to inform 
users about wheelchair accessibility in urban areas \cite{Appleyard2005}



\section{Big data and Cloud Computing for Disability Informatics}
The contribution of Big data and cloud computing have been recognized and accepted by researchers in health informatics \cite{7047725}. The potential of big data and cloud computing for disability informatics and for people with disabilities has 
been explored by a few researchers and organizations. Data-Pop Alliance is one of the organization has recognized 
the big data and potential for study and help people with disabilities for disability informatics and people 
with disabilities \cite{Datapop}. Their research has categorized three type of big data source used across disability
research: exhaust data (mobile-based data, financial transaction, transportation and online trace), digital 
content (social media and crowed-sourced/online content), and sensing data (physical and remote) \cite{Datapop}. 
They also provided the potential for some of these data sources, for example, researchers can use transaction 
data to compare cost, availability, and use of services that offer accessible options (such as accessible hotel 
listings) \cite{Datapop}. They also suggested that researcher can use social media data to represent people with
disabilities as a network of interaction and using crow-sourcing to map the locations of accessible businesses 
and public places \cite{Datapop}. 
The organization has also identify four functions of big data on disability:descriptive, predictive, diagnostic 
and engagement. Descriptive function of big data is to describing and presenting the collected information such
as using location data to map workplaces that are accessible to people with disabilities \cite{Datapop}. 
Predictive function is making inferences based on collected information such as discovering trends in the 
growth of number of accessible businesses in a certain urban area, while the diagnostic function means
establishing and making recommendations on the basis of causal relations such as showing what can help 
increasing accessible business in a certain area \cite{Datapop}. Finally, the engagement function refer
to shaping dialogue within and between communities and with key stakeholders through communication of data \cite{Datapop}.


Cloud computing in combined with big data can also provide great opportunities for research and improvement of
quality of life for people with disabilities \cite{Caldwell2011}. The term cloud ``refers to everything a user may reach via the 
Internet, including services, storage, applications, and people'' \cite{Hoehl2010}. Depending on the type of 
using, the ``cloud'' can be use for different purpose, such as for companies, the cloud could be used for hosting
services so as to avoid the costs and difficulties associated with hosting one’s own servers and software and for
individuals, the could is often used as information storage \cite{Khazaei14}. Regardless of the types of usages 
for cloud, the end using must still access the information and services residing in the cloud through device like
a smart phone or computer \cite{Hoehl2010}. Cloud computing has been used to provide more accessible virtual 
environment, especially Web access through project like WebAnywhere, which is a cloud based tool for blind using 
to access Internet \cite{Hoehl2010}. 

Cloud computing and big data analytics can also be helpful in health monitoring. The Artemis project mention earlier
provide a example of big data analytics and cloud computing usage in health monitoring, by creating new cloud-base
health analytics solutions \cite{Khazaei14}. Previous researchers have developed a mobile app to collect motion 
data of Parkinson's disease (PD) which is a disease resulting in mobility disorder using the smart phone 3D 
accelerometer and to send the data to a cloud service for storage, data processing, and PD symptoms severity
estimation, which provide an user-friendly and economically affordable system to monitor and assess the condition
of PD \cite{info:doi/10.2196/mhealth.3956}. Although this system is not for people with disabilities, but it 
provided potentials for similar systems to be developed for different kind of disabilities. 

Another application of cloud computing and big data in assistive technology is the CloudCast platform, which is a 
cloud-based speech recognition services that can be used for many assistive technology application for people
with speech difficulties and hearing impairment, it also facilitate the collection of speech data required for
the machine learning techniques \cite{cunningham2017cloud}. Similar to Alexa Voice Service, it provide reliable 
speech recognition which can be used with assistive devices for people with hearing impairments, but CloudCast
platform also provide customization for assistive technology applications benefiting users with speech 
impairment \cite{cunningham2017cloud}. This research provided a great example of using big data and cloud 
computing in combine to solve a certain problem for people with disabilities (in this case it is barriers for speech impairment). 


\section{Conclusion}
People with disabilities has long been considered as underprivileged groups. Although the development of information
technology and assistive technology has improved the life of people with disabilities, there is still much left to
done. This study provided a short overview of disability informatics, providing examples of using health informatics
for people with disabilities. This study has also find out there are great potentials to use big data source to
better represent people with disability and identity and study issues and propose actions and solution to the
challenges faced by people with disabilities. Cloud computing and big data can also help improving assistive and
information technology that are now used to help people with disabilities.
However, there are still a lot challenge faced by researchers and organizations interested in improving the quality
of life for people with disabilities. The most dominated challenge is the different needs for people with different
disabilities types and function levels. 

The limitation of this study is that there are limited number of studies on disability informatics or big data. This
study draws some of the examples from health informatics which their study focuses were other disease. However, these
examples do show opportunities to developing similar systems for special needs of people with disabilities. Another
limitation is that also there are organizations that have illustrated the potentials to use big data in analyzing and
understanding people with disabilities, there are not yet studies has been done to prove these potentials. However,
this leaves opportunities for future researchers to use big data to better understand the needs and behaviors of
people with disabilities. 


\begin{acks}

  The author would like to thank Dr. Gregor von Laszewski for his
  support and suggestions to write this paper.

\end{acks}

\bibliographystyle{ACM-Reference-Format}
\bibliography{report} 

\appendix


\end{document}
