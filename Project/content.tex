\title{Big Data Applications in the Travel Industry and its Potential in Improving Travel Accessibility}


\author{Weixuan Wang}
\affiliation{%
  \institution{Indiana University Bloomington}
  \streetaddress{School of Public Health}
  \city{Bloomington} 
  \state{Indiana} 
  \postcode{47405}
}
\email{wangweix@indiana.edu}


% The default list of authors is too long for headers}
\renewcommand{\shortauthors}{Weixuan Wang}


\begin{abstract}
This is my abstract
\end{abstract}

\keywords{i523, HID234}


\maketitle



\section{Introduction}
People with disabilities represented a large neglected tourism market. According to
Amadeus annual report, 15 percent of worldwide population (1 billion people) lives
with some form of disability \cite{Ama}. According to the Open Door Organization (ODO) market report in 2015, people with disabilities spend 17.3 billion dollars annually for their own travel. Because people with disabilities usually needs a care giver or family member to accompany them when traveling, the potential economic impact could double \cite{ODO}. 
Travel Industry and its Potential in Improving Travel Accessibility.
Here is the introduction. Travel accessibility or travel for people with disabilities. Aspects that
need to consider are transports, hotel,   \cite{Ama}.

\section{transportation and big data}

\section{Travel and Big data}

\section{Promise of big data in travel accessibility}

\subsection{Sentiment Analysis on Hotel reviews}
Sentiment analysis, which is also called opinion mining, is one of the most active
research areas in natural language processing \cite{opinion2014}. The aim of sentiment analysis is to define automatic tools able to extract subjective information from text in natural \cite{article} language, and to create structured and actionable knowledge to be used by either a decision support system or a decision maker. The sentiments of reviews, online reputation or online documents are usually categorized in positive, negative and (in some studies) neutral sentiments \cite{Garcia2012}. The main goal of the sentiment classification is to extract ``the global sentiment based on the subjectivity and the linguistic characteristics of the words within an unstructured text'' \cite{Garcia2012}. Therefore, sentiment analysis provided a framework to transform unstructured text to structured data, which make it strongly applicable to both the academic field \cite{Cam2013}. Because of the importance of sentiment analysis to business and society, it has spread from computer science to management science and the social sciences \cite{Pozzi}. As a social science field and business industry, tourism an travel studies have already been using sentiment analysis in the research.

The popularity of social media, especially review sites like TripAdvisor and blogs and wikis, leads to an enormous amount of personal reviews for travel-related information on the Web \cite{opinion2014}. More importantly, the information in these reviews is valuable to both tourists and travel and tourism practitioners for various understanding and planning processes \cite{YE20096527}. Since 2009, researchers have been using machine learning methods in the natural language processing (support vector machine (SVM), Naïve Bayes, and the N-gram model) to do sentiment analysis on TripAdvisor reviews \cite{YE20096527}. Their study analyzed online reviews related to travel destinations, using different supervised machine learning algorithms The
algorithms to evaluate the reviews about seven popular travel destinations in Europe and North America \cite{YE20096527}. 

In tourism studies, sentiment analysis has been compared to traditional qualitative analytic methods. A previous study compared three alternative approaches for mining consumer sentiment (manual content coding, corpus-based semantic analysis, and stance-shift analysis) from large amounts of qualitative data found in online travel reviews.  They applied three different approaches to study consumers' reaction to farm stays in order to demonstrate how large volumes of qualitative data can be analyzed quantitatively in a relatively efficient and reliable way \cite{Farm}. By comparing the three approaches of sentiment analysis, 


\subsection{Recommender System}
Information Communication Technologies (ICTs) have been transforming tourism business
globally and revolutionizing the world of Tourism. It transforms tourism from a
labor-intensive to an information-intensive industry \cite{Williams201787}.
Tourists influence by the developments in search engines, network speed and capacity 
have been using use technologies for better planning and experiencing their
trips \cite{XIE2017101}. In addition, ICTs enable travelers to access reliable and
accurate information and make reservations faster, cheaper and more convenient than
the traditional way \cite{chung2009}.

The development of ICTs also enables Internet users to both create and distribute
information (especially multimedia information), which is called user-generated 
content (UGC) or consumer-generated content (CGC) \cite{chung2009}.

Nowadays tourists faces a very challenging task of trip preparation because of the
huge amount of information available on the Web about tourism and leisure activities.
Recommender systems becomes essential for tourists and tourism operates. For tourists,
recommender systems can be a useful tools to help them make decision for travel
planning, such as the choices of destinations, attractions, accommodations and
restaurants. As for tourism operators, it can be a great marketing opportunities for
them to reach a variety of targeted potential consumers. Complex problems such as
automated planning, semantic knowledge management, group recommendation or
context-awareness have by now been heavily studied in this area \cite{morenorecommender}. 

There are already several tourism recommender system available for general public. TIP and Heracles systems provide recommendation service through mobile devices for tourism, through implement hybrid algorithms to calculate tourist preferences, using the defined tourist profile and location data \cite{morenorecommender}. Crumpet provides new information delivery services for a variety of different tourist population based on location aware services, personalized user interaction, accessible multi-media mobile communication and smart component-based middleware that uses Multi-Agent Technology \cite{Santos2018}. CATIS is a Web based tourist information system using context-awareness, which include context elements such as location, time of day, speed, direction of travel and personal preferences. This system provided information to tourists relevant to his or her location and time \cite{Santos2018}. TravelWithFriends using group recommendation service, the first step is to build a recommendation list for each user and to merge them to obtain a destinations shortlist. Afterwards, each group member rates all these options and a Borda count is used to determine the best five destinations to be recommended \cite{morenorecommender}.

Classical recommender systems filter the domain items according to a particular user, using his or her demographic data, past ratings or purchasing history \cite{LU201512}. This approach are used to recommend specific items such as books, songs or films \cite{LU201512}. However, it may not be suitable for travel activities, since most of time travel is an activity that involves a group of people (such as family members, friends). Therefore, it is necessary to take into account the different preferences of all members of travel group when providing recommendations \cite{morenorecommender}. Previous studies and technology reports have identified two primary options for group recommendation: the first one is to merge the lists of items recommended to each group member, or creating a group profile with everyone’s preferences and then compute a single list of group recommendations \cite{Garcia2009}. The second option’s first step is the same as the previous option, by constructing of a list of recommendations for each group member. In a second step though, an automatic consensus-reaching process is applied, in which individual preferences are continuously updated until a high degree of agreement between all the group members is reached \cite{Garcia2009}.

The use of semantic domain knowledge in the recommendation process has heavily increased in recent years. Previous studies have defined the semantic similarity between two concepts is as ``the ratio between the number of different ancestors and the total number of ancestors of both concepts’’ \cite{morenorecommender}. The items to be recommended are clustered according to this semantic similarity and the recommendation procedure selects the best item from random clusters \cite{Santos2018}. Previous study has shown that this procedure keeps the accuracy and increases the diversity of the results \cite{morenorecommender}. Semantic information can also be used to determine the items to be recommended in a personalized visit to a museum or destinations, by using a shortest-path semantic distance to determine which museum objects or attractions should be recommended to the user \cite{morenorecommender}. 

Previous study also proposed a hybrid tourism recommendation system for persons suffering from physical or intellectual limitation. This proposed recommendation system is not simply trying to improve experience, but to create in users the confidence that despite of their limitations they can visit and experience certain places without being afraid, and to help them to live a touristic experience. By identifying the user’s functionality and point of interest (POI) accessibility level, the system models a user stereotype profile, which represent user’s related knowledge (which is layered with several knowledge representation structures and models shown in Fig. 1) and produce an accurate touristic recommendation plan \cite{Santos2018}.



\section{Conclusion}

Put here an conclusion. Conlcusions and abstracts must not have any
citations in the section.


\begin{acks}

  The authors would like to thank Dr. Gregor von Laszewski for his
  support and suggestions to write this paper.

\end{acks}

\bibliographystyle{ACM-Reference-Format}
\bibliography{report} 

\appendix
