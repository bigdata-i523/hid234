\title{Big data Analytics and Applications in the Travel Industry and its Potential in Improving Travel Accessibility}


\author{Weixuan Wang}
\affiliation{%
  \institution{Indiana University Bloomington}
  \city{Bloomington} 
  \state{Indiana} 
  \postcode{47405}
}
\email{wangweix@indiana.edu}
 

% The default list of authors is too long for headers}
\renewcommand{\shortauthors}{Weixuan Wang}


\begin{abstract}
Big data applications and analytics have been influencing and improving tourists' experience. Travel accessibility refers to provide access for people with disabilities or limited mobility (such as seniors), who represent a growing market in the travel industry by spending billions on leisure and business trips. This report explored the implementation of big data analytics and applications in tourism, disability informatics and assistive technologies for people with disabilities. This report explore the potentials of big data applications and analytics in understanding travel experience of people with disabilities and improving travel accessibility 
\end{abstract}

\keywords{i523, HID234, Big Data Analytics, Travel Accessibility, People with Disabilities}


\maketitle



\section{Introduction}
People with disabilities represented a large neglected tourism market. According to
Amadeus annual report, 15 percent of worldwide population (around 1 billion people) lives
with some form of disability \cite{Ama}. According to United Nation, people with 
disabilities are the largest minority group in the world
\cite{Appleyard2005,DARCY2010816,Lex}. Notably, the number of people with disabilities is
expected to increase as a  result of extension of human life-span, decreases in
communicable diseases, the improvement of medical technology, and decrease of child
mortality \cite{SMITH1987376}.  While some forms of disabilities might be genetic, but
temporary or permanent disabilities can happen to anyone, such 
as spinal cord injury after car accident, or limited mobility at later stage of life
\cite{Lex}. 

Population aging trend also signifies that disability will be a more common
and urgent issue in the future \cite{Grue}. The World Health Organization estimates that
by 2050, 21.5 per cent of the global population will be aged over 65. People with
accessibility issues represent a large and fast growing minority group
worldwide, and travel and tourism demand of this group is often ignored. According to the
Open Door Organization (ODO) market report in 2015, people with disabilities spend 17.3
billion dollars annually for their own travel. Because people with disabilities usually
needs a care giver or family member to accompany them when traveling, the potential
economic impact could double \cite{ODO}. 


Accessible travel or accessible tourism refers to the inclusive travel activities that
enable people with access requirements, including mobility, vision, hearing and cognitive
dimensions of access, to function independently, with equity and dignity through the
delivery of universally designed tourism products, services and environments \cite{Ama}.
However, the travel experiences for people with disabilities are more than access
issues. In order to achieve travel accessibility, which means provide travel activities
for people with disabilities, a variety of aspects for travel needs to be taken in
consideration. An accessible destination and appropriate accommodation only lay the foundation for a particular travel experience to happen for people with disabilities \cite{ODO}. More aspects that need to
consider for people who are traveling with disabilities, such as accessible
transportation, accessible online booking\cite{Ama}.

The ultimate aim for those involved in supporting accessible travel is to empower every
individual to plan and travel independently, at their own will \cite{zhang2016}. However,
the task is not a easy one. Making the whole travel chain accessible, including the
information and booking procedures, as well as the infrastructure and processes become a
important task for travel accessibility \cite{Ama}. 

With the development of information and digital technology has changed many people's
lives, especially the life of people with disabilities has also been improved by
technology \cite{GJT14}. People with poor visions can using cell phones to contact
others, access information online with screen readers. People with hearing problems can
text other people with their cell phone. 

The development of information communication technologies especially the creation
and distribution of user-generated content (UGC) or consumer-generated content (CGC) has
successfully changed how people travel and how people gather information for travel
\cite{chung2009}. Big data application and analytics has become a trending topic for both
the tourism industry and tourism studies \cite{chung2009}. The use of big data for
disability informatics and developing assistive technology has been studies to improve
the quality of life for people with disabilities \cite{Grue}. 

\section{Tourism and Big data}
Most activities in the tourism industry had been generating a huge amount of data for
several years. Booking flight tickets, reserving a hotel room and renting a car all
leaves a data trail \cite{Shafiee16}. These data could add up to more than hundred of
terabytes or petabytes structured data in the conventional databases \cite{akerkar2012}.
Discussions of travel planning on online travel community, status updates and posts on
social media like Facebook and Twitter, compliments and compliant on review websites like TripAdvisor
constructs more challenging and live unstructured data that arrives at a much
faster pace than a conventional database \cite{akerkar2012}. Tourism practitioners are
trying to understand tourists' behavior by accepting and analyzing these big data 
\cite{Shafiee16}.

Tourists in the digital age often use a variety of tools to access information that
the tourism industry or other users have provided \cite{XIANG2015120}. A tourist
produces a high volume of data when they are searching for travel websites, reporting
issues on mobile applications, sharing traffic information in the cities, searching and
posting on social media, taking and sharing photos, reporting experience on travel
websites and social media \cite{akerkar2012, Shafiee16}. All these data that are
produced constantly can demonstrate tourists' motivation, interests, and their
planning patterns and so on \cite{XIE2017101}.

Previous studies have demonstrated several different usage and formats of big data in
the travel and tourism industry \cite{XIE2017101}. Social media is one of them that
has a huge effect on the tourism industry. Social media includes social networks,
review sites, blogs, media sharing, and wikis \cite{XIANG2015120}. The exceptional
growth of these data sources has inspired companies and institutions to come up with 
new strategies to understand the socio-economic phenomenon in various
fields \cite{Shafiee16}. Discussions and information sharing on social media are
considered as electronic word-of-mouth (eWOM) that has in some degree substituted
tradition face-to-face word-of-mouth (WOM) for information exchange of tourist
experience \cite{chung2009}. 

\section{Disability and Big Data}
Disability informatics is a sub-specialty of health informatics that is defined as ``any application that
collects, manages, and distributes information that are related to people with disabilities, as well as to
care givers (including familiar members and health care providers) and rehabilitation professionals''
\cite{Appleyard2005}. Disability informatics is closely related to other health informatics areas such as
medical informatics, public health informatics and consumer health informatics, because people with
disabilities usually have some secondary medical condition such as poor health status and increased
personal health care needs. 

Gather medical and health information can help to better understand and
accommodate people with disabilities \cite{Riga13}. A study from the early 2000 has identified the
potential of public health informatics for prevention at all vulnerable points in the causal chains leading
to disability and proposed that  applications should not be restricted to particular social, behavioral,
or environmental contexts, but in a more global context \cite{Yasnoff}.  

Another previous research has
designed and deployed an extended version of Artemis system (a cloud system designed to acquire data and
store physiological data of clinical information for real-time analytics) in a hospital. They have
identified that high speed physiological data produced at intensive care units as big data, and the proper
use of such data can promote health, reduce mortality and disability rates of critical condition patients
and create new cloud-based health analytics \cite{Khazaei14}. Research also has shown that many
disabilities are genetic, therefore, bioinformatics has implications in the education of genetic screening
and gen therapy treatments in the future \cite{Appleyard2005}.

People with disabilities usually need some 
assistive technology in their daily life. These technology that assist them to perform basic physical and 
social functions. The use information technology and assistive applications in disability informatics are 
categorized into three areas: virtual, personal, physical.
 
 \subsection{Virtual Environment}
The digital revolution had and will continue to have a profound positive impact on the life of people with
disability by empowering them with the help of digital technologies \cite{Appleyard2005}. However, there are
still access issues in the digital world. One of the barrier is the use of the World Wide Web (WWW or Web).
The WWW has always had a strong awareness and been advocacy for accessibility since early on in its
evolution. The World Wide Web Consortium (W3C) had passed the Web Accessibility Initiative (WAI) and Web
Content Accessibility Guidelines in the late 1990s \cite{Appleyard2005}. A number of assistive technologies
were designed to help people with disabilities to use the Web. For example BBC Education Text to Speech
Internet Enhancer (BESTIE) is a CGI Perl script that can help people with disabilities who are using
text-to-speech systems for Web browsing to modified the web page removing images, Java and Javascript code
that may cause difficulties to understand the BBC web page content \cite{Erra}. 
However, the limitation of BESTIE is that it is only compatible with BBC website. Other researchers also
came up with Personalizable Accessible Navigation (PAN), which is a set of edge services designed to improve
Web pages accessibility which allow  personalization and the opportunities to select multiple profiles,
making it compatible for web as well as mobile devices \cite{info:doi/10.2196/mhealth.3956}.

\subsection{Personal Environment}
 Disability informatics also emphasis on providing safe personal environment for people with disabilities,
 Health monitoring is a very important area. Technologies like small tracking device can monitor heart rate,
 blood pressure, also allow people to call for help easily and smart clothing and even smart furniture have
 been developed to monitor people's health status and can help provide people with disabilities a safer
 personal environment and also provide health information for their medical care providers
 \cite{Appleyard2005}. However the ethic of such health monitor devices are always in debate, some believe
 it can be an invasion of privacy and a restriction of personal freedom, others hold the ground that its
 main purpose is to help people with disease or disabilities, since it can alert their caregiver if the
 individual are exposed to harm (such as a person with mental disability and has a history of self-harming,
 these device can prevent unwanted behavior) \cite{cunningham2017cloud}. 
 

\subsection{Physical Environment}
Since the American Disabilities Act passed in the 1990s, the accessibility of physical environment has been
improved in a great degree. However, people with disabilities still would meet some barrier and problem, 
one of them is the lack of curb cuts. Assistive information technologies has been developed in an effort to
solve this problem. One of them is MAGUS, which is a project using geographical information system to inform 
users about wheelchair accessibility in urban areas \cite{Appleyard2005}


 The contribution of Big data and cloud computing have been recognized and accepted by researchers in health informatics \cite{7047725}. The potential of big data and cloud computing for disability informatics and for people with disabilities has 
been explored by a few researchers and organizations. Data-Pop Alliance is one of the organization has recognized 
the big data and potential for study and help people with disabilities for disability informatics and people 
with disabilities \cite{Datapop}. Their research has categorized three type of big data source used across disability
research: exhaust data (mobile-based data, financial transaction, transportation and online trace), digital 
content (social media and crowed-sourced/online content), and sensing data (physical and remote) \cite{Datapop}. 
They also provided the potential for some of these data sources, for example, researchers can use transaction 
data to compare cost, availability, and use of services that offer accessible options (such as accessible hotel 
listings) \cite{Datapop}. They also suggested that researcher can use social media data to represent people with
disabilities as a network of interaction and using crow-sourcing to map the locations of accessible businesses 
and public places \cite{Datapop}. 
The organization has also identify four functions of big data on disability:descriptive, predictive, diagnostic 
and engagement. Descriptive function of big data is to describing and presenting the collected information such
as using location data to map workplaces that are accessible to people with disabilities \cite{Datapop}. 
Predictive function is making inferences based on collected information such as discovering trends in the 
growth of number of accessible businesses in a certain urban area, while the diagnostic function means
establishing and making recommendations on the basis of causal relations such as showing what can help 
increasing accessible business in a certain area \cite{Datapop}. Finally, the engagement function refer
to shaping dialogue within and between communities and with key stakeholders through communication of data \cite{Datapop}.


Cloud computing in combined with big data can also provide great opportunities for research and improvement of
quality of life for people with disabilities \cite{Caldwell2011}. The term cloud ``refers to everything a user may reach via the 
Internet, including services, storage, applications, and people'' \cite{Hoehl2010}. Depending on the type of 
using, the ``cloud'' can be use for different purpose, such as for companies, the cloud could be used for hosting
services so as to avoid the costs and difficulties associated with hosting one’s own servers and software and for
individuals, the could is often used as information storage \cite{Khazaei14}. Regardless of the types of usages 
for cloud, the end using must still access the information and services residing in the cloud through device like
a smart phone or computer \cite{Hoehl2010}. Cloud computing has been used to provide more accessible virtual 
environment, especially Web access through project like WebAnywhere, which is a cloud based tool for blind using 
to access Internet \cite{Hoehl2010}. 

Cloud computing and big data analytics can also be helpful in health monitoring. The Artemis project mention earlier
provide a example of big data analytics and cloud computing usage in health monitoring, by creating new cloud-base
health analytics solutions \cite{Khazaei14}. Previous researchers have developed a mobile app to collect motion 
data of Parkinson's disease (PD) which is a disease resulting in mobility disorder using the smart phone 3D 
accelerometer and to send the data to a cloud service for storage, data processing, and PD symptoms severity
estimation, which provide an user-friendly and economically affordable system to monitor and assess the condition
of PD \cite{info:doi/10.2196/mhealth.3956}. Although this system is not for people with disabilities, but it 
provided potentials for similar systems to be developed for different kind of disabilities. 

Another application of cloud computing and big data in assistive technology is the CloudCast platform, which is a 
cloud-based speech recognition services that can be used for many assistive technology application for people
with speech difficulties and hearing impairment, it also facilitate the collection of speech data required for
the machine learning techniques \cite{cunningham2017cloud}. Similar to Alexa Voice Service, it provide reliable 
speech recognition which can be used with assistive devices for people with hearing impairments, but CloudCast
platform also provide customization for assistive technology applications benefiting users with speech 
impairment \cite{cunningham2017cloud}. This research provided a great example of using big data and cloud 
computing in combine to solve a certain problem for people with disabilities (in this case it is barriers for speech impairment). 

\section{Accessible Transportation and Big data}

An inaccessible transport network prevents many people from going to school or studying, working, going
to the doctor, meeting friends, going shopping or to the cinema and other activities that
are taken for granted, especially for people with disabilities, an inaccessible transport
network would left them dependent and confined in their own home \cite{Ama}. A
inaccessible transport network would, therefore, also prevent people with disabilities to
travel for business or leisure \cite{milo}.
Older adults, people with disabilities, individuals in low-income households, especially 
those living in rural areas can face significant mobility challenges \cite{Farm}. 

Concerns about getting into an accident, congestion,
price of travel, access to transit, and lack of walkways are important issues for a large
percentage of the population, but they tend to be more important for people with
disabilities \cite{moya2016dynamic}. For today's travel and transportation businesses,
it is important to address the issue of inclusion, which is the potential to enable 
a broader range of people to use
transportation infrastructure regardless of their individual abilities or disabilities
\cite{milo}. 

Accessible travel includes not only the point-to-point transportation (such as air
travel, flights), but also the accessibility of destination \cite{Ama,DARCY2010816,milo}.
For people with disability to actually make the trip, they will also require booking for
transportation to be accessible. This section is going to explore the implementation of big
data analytics and applications in air travel, online booking and destination
accessibility. 

\subsection{Big data Analytics and Applications for Airline and Online Reservation}

\subsubsection{Airline and Big data}

The airline industry is very familiar with big data use in their daily operations and market research. Airline companies have been using their big data which is the large volume of structured information that has been produced internally \cite{MIAH2017} to analyze prices of plane ticket. Moreover, airlines have optimized the details of planning for the crew and routing \cite{Shafiee16}. 

Previous studies in airline network used Big Data mined from the U.S. air transportation system over the years from 1998–2014 to characterize the network's behavior and determine what internal and/or external drivers result in structural changes to the airline network \cite{7777957}. Airline delay patents has also been studied with the help of big data by identifying by the number of late arrivals as a percent of total operations \cite{Sor}. 

In another previous study, researchers used data from 2006 to 2008 in order to provides the result about the total flight delay for a specific period of time caused due to climate, security, carrier, National Aviation System, Arrival and Departure based on total number of flights getting delayed over in the given period of time \cite{Sor}. In the study, the authors used time series analysis along with the integration of heterogeneous database to identify and achieve the Airline Seasonal Delay which is implemented and visualized in R, they were able to identify a trend line to provide the insights for the aviation industry to take future measures to avoid delays and manage them \cite{Sor}.

Airline studies have also used big data analytics on passenger reviews data.The advance development of social media and mobile helped the passengers to post reviews in a ubiquitous way, allow them post real time feedback over Facebook, Twitter on airports, airlines, and other travel providers \cite{CHEN2016285}. However, passengers' review can be really complex since travel activities usually involve multiple parties, therefore, the travel domain application systems are also typically managed by different stakeholders like airlines, airports, travel agencies, security and other services providers like cars, bus, trains, hotels, events. In order to provide a holistic approach to manage complex passenger reviews with data gathering, processing and disseminating, a previous study has proposed a reference architecture to manag passenger reviews where multiple stakeholders are involved by using data lakes, which can store, manage and analyze structured and unstructured data with cheaper cost, well-distributed, open sourced and powerful set of tools.

\subsubsection{Big Data Applications and Online Reservation}
The online sector of the tourism industry has also
quickly adopted big data to improve internal decisions and understand customers 
\cite{akerkar2012}. The online sector of the industry include  meta-search engines (like 
Google), online travel agencies (like Expedia) and some information website companies that
distribute tourism information (TripAdvisor)\cite{MIAH2017}. For example, Amadeus has 
developed a program ``Amadeus Airline Cloud Availability'' that can generated special 
result and increase search for its customers and Kayak has developed a program to predict 
costs and prices for tourists\cite{Shafiee16}.

The digital revolution had and will continue to have a profound positive impact on the life of people with
disability by empowering them with the help of digital technologies \cite{Appleyard2005}. However, there are
still access issues in the digital world. One of the barrier is the use of the World Wide Web (WWW or Web).
The WWW has always had a strong awareness and been advocacy for accessibility since early on in its
evolution. The World Wide Web Consortium (W3C) had passed the Web Accessibility Initiative (WAI) and Web
Content Accessibility Guidelines in the late 1990s \cite{Appleyard2005}. A number of assistive technologies
were designed to help people with disabilities to use the Web. For example BBC Education Text to Speech
Internet Enhancer (BESTIE) is a CGI Perl script that can help people with disabilities who are using
text-to-speech systems for Web browsing to modified the web page removing images, Java and Javascript code
that may cause difficulties to understand the BBC web page content \cite{Erra}. 
However, the limitation of BESTIE is that it is only compatible with BBC website. Other researchers also
came up with Personalizable Accessible Navigation (PAN), which is a set of edge services designed to improve
Web pages accessibility which allow  personalization and the opportunities to select multiple profiles,
making it compatible for web as well as mobile devices \cite{info:doi/10.2196/mhealth.3956}.
Travel domain companies like Marriott, Southwest airline, and Amtrak also developed assistive devices specially for people with disabilities to use when browsing their websites. 
These assistive technologies can help people with disabilities to navigate online reservation website, and help them to independently booking their travel reservations for hotels, restaurants, airplane tickets and attraction passes.


\subsection{Destination Accessibility and Big data}
Accessibility has always been a key concept in urban and regional planning for its capacity to link the activities of people and businesses to the possibilities of reaching them effectively. The concept of urban accessibility are different from travel accessibility, since the urban accessibility is focus on provide access of transportation and transit to general public, not specifically people with disabilities \cite{moya2016dynamic}. However, previous studies using big data on urban accessibility can provide some potentials for travel accessibility studies.




\section{Promise of big data in travel accessibility}

\subsection{Sentiment Analysis on Hotel reviews}
Sentiment analysis, which is also called opinion mining, is one of the most active
research areas in natural language processing \cite{opinion2014}. The aim of sentiment analysis is to define automatic tools able to extract subjective information from text in natural \cite{article} language, and to create structured and actionable knowledge to be used by either a decision support system or a decision maker. The sentiments of reviews, online reputation or online documents are usually categorized in positive, negative and (in some studies) neutral sentiments \cite{Garcia2012}. The main goal of the sentiment classification is to extract ``the global sentiment based on the subjectivity and the linguistic characteristics of the words within an unstructured text'' \cite{Garcia2012}. Therefore, sentiment analysis provided a framework to transform unstructured text to structured data, which make it strongly applicable to both the academic field \cite{Cam2013}. Because of the importance of sentiment analysis to business and society, it has spread from computer science to management science and the social sciences \cite{Pozzi}. As a social science field and business industry, tourism an travel studies have already been using sentiment analysis in the research.

The popularity of social media, especially review sites like TripAdvisor and blogs and wikis, leads to an enormous amount of personal reviews for travel-related information on the Web \cite{opinion2014}. More importantly, the information in these reviews is valuable to both tourists and travel and tourism practitioners for various understanding and planning processes \cite{YE20096527}. Previous studies have identified two primary approaches for sentiment analysis: methods based on the combination of lexical resources and Natural Language Processing (NLP) techniques; and machine learning approaches \cite{Garcia2012}. Since 2009, researchers have been using machine learning methods in the natural language processing (support vector machine (SVM), Naïve Bayes, and the N-gram model) to do sentiment analysis on TripAdvisor reviews \cite{YE20096527}. 
Their study analyzed online reviews related to travel destinations, using different supervised machine learning algorithms The
algorithms to evaluate the reviews about seven popular travel destinations in Europe and North America \cite{YE20096527}.

Thee etBlogAnalysis project developed a combined crawler /sentiment extraction application for the tourism industry, which used a simple and robust linguistic parsing methodology with information and terminology extraction methods in order to determine relevant utterances on expression level \cite{opinion2014}. It will also provide a warning for tourism operator such as a hotel, if too many negative entries have been generated by their reviewers \cite{Garcia2012}.

In tourism studies, sentiment analysis has been compared to traditional qualitative analytic methods. A previous study compared three alternative approaches for mining consumer sentiment (manual content coding, corpus-based semantic analysis, and stance-shift analysis) from large amounts of qualitative data found in online travel reviews \cite{Farm}. They applied three different approaches to study consumers' reaction to farm stays in order to demonstrate how large volumes of qualitative data can be analyzed quantitatively in a relatively efficient and reliable way \cite{Farm}.  Manual content coding is the same as traditional the content analysis approach involving two researchers collaborated in a manual coding process designed to extract consumer likes and dislikes from the qualitative data \cite{Farm}. According to the comparison, computer generated sentiment analysis such as stance-shift analysis processing on both syntax and lexicon assures the coding maintains the statement's context identifying what is important to the informants by the way they express their comments. Most importantly, stance-shift analysis does not categorize what the researcher thinks is important in reviewer's words \cite{Farm}. The study suggested by combining different approaches in sentiment analysis such as using stance-shift analysis first identifies the significant word segments then using corpus-based semantic analysis detects key themes in those segments helps uncover narrative themes of consumer experiences in large qualitative databases \cite{Farm}.

Sentiment analysis will help researchers to better understand people's travel experience, however, there are few studies have been done to identify demographic information of the reviewer and compare the sentiment analysis result across different demographic. A recent invention present the possibility of identifying demographic characteristics while conducting sentiment analysis. The invention consist of a product or service review to determine demographic information of the reviewer \cite{Bhatt2014}. A sentiment text analysis is performed on the product or service review, wherein the sentiment text analysis examines the product or service review to determine a sentiment of the product or service review. The sentiment of the product or service review is categorized based on the demographic information of the reviewer \cite{Bhatt2014}. This invention present the promise of using sentiment analysis on the travel experience of people with disabilities. However, challenges still remain for research of UGC generated by people with disabilities, such as the challenge presented by privacy concerns of personal data online \cite{lazar}. 


\subsection{Recommender System}
Information Communication Technologies (ICTs) have been transforming tourism business
globally and revolutionizing the world of Tourism. It transforms tourism from a
labor-intensive to an information-intensive industry \cite{Williams201787}.
Tourists influence by the developments in search engines, network speed and capacity 
have been using use technologies for better planning and experiencing their
trips \cite{XIE2017101}. In addition, ICTs enable travelers to access reliable and
accurate information and make reservations faster, cheaper and more convenient than
the traditional way \cite{chung2009}.

The development of ICTs also enables Internet users to both create and distribute
information (especially multimedia information), which is called user-generated 
content (UGC) or consumer-generated content (CGC) \cite{chung2009}.

Nowadays tourists faces a very challenging task of trip preparation because of the
huge amount of information available on the Web about tourism and leisure activities.
Recommender systems becomes essential for tourists and tourism operates. For tourists,
recommender systems can be a useful tools to help them make decision for travel
planning, such as the choices of destinations, attractions, accommodations and
restaurants. As for tourism operators, it can be a great marketing opportunities for
them to reach a variety of targeted potential consumers. Complex problems such as
automated planning, semantic knowledge management, group recommendation or
context-awareness have by now been heavily studied in this area \cite{morenorecommender}. 

There are already several tourism recommender system available for general public. TIP and Heracles systems provide recommendation service through mobile devices for tourism, through implement hybrid algorithms to calculate tourist preferences, using the defined tourist profile and location data \cite{morenorecommender}. Crumpet provides new information delivery services for a variety of different tourist population based on location aware services, personalized user interaction, accessible multi-media mobile communication and smart component-based middleware that uses Multi-Agent Technology \cite{Santos2018}. CATIS is a Web based tourist information system using context-awareness, which include context elements such as location, time of day, speed, direction of travel and personal preferences. This system provided information to tourists relevant to his or her location and time \cite{Santos2018}. TravelWithFriends using group recommendation service, the first step is to build a recommendation list for each user and to merge them to obtain a destinations shortlist. Afterwards, each group member rates all these options and a Borda count is used to determine the best five destinations to be recommended \cite{morenorecommender}.

Classical recommender systems filter the domain items according to a particular user, using his or her demographic data, past ratings or purchasing history \cite{LU201512}. This approach are used to recommend specific items such as books, songs or films \cite{LU201512}. However, it may not be suitable for travel activities, since most of time travel is an activity that involves a group of people (such as family members, friends). Therefore, it is necessary to take into account the different preferences of all members of travel group when providing recommendations \cite{morenorecommender}. Previous studies and technology reports have identified two primary options for group recommendation: the first one is to merge the lists of items recommended to each group member, or creating a group profile with everyone’s preferences and then compute a single list of group recommendations \cite{Garcia2009}. The second option’s first step is the same as the previous option, by constructing of a list of recommendations for each group member. In a second step though, an automatic consensus-reaching process is applied, in which individual preferences are continuously updated until a high degree of agreement between all the group members is reached \cite{Garcia2009}.

The use of semantic domain knowledge in the recommendation process has heavily increased in recent years. Previous studies have defined the semantic similarity between two concepts is as ``the ratio between the number of different ancestors and the total number of ancestors of both concepts’’ \cite{morenorecommender}. The items to be recommended are clustered according to this semantic similarity and the recommendation procedure selects the best item from random clusters \cite{Santos2018}. Previous study has shown that this procedure keeps the accuracy and increases the diversity of the results \cite{morenorecommender}. Semantic information can also be used to determine the items to be recommended in a personalized visit to a museum or destinations, by using a shortest-path semantic distance to determine which museum objects or attractions should be recommended to the user \cite{morenorecommender}. 

Previous study also proposed a hybrid tourism recommendation system for persons suffering from physical or intellectual limitation. This proposed recommendation system is not simply trying to improve experience, but to create in users the confidence that despite of their limitations they can visit and experience certain places without being afraid, and to help them to live a touristic experience. By identifying the user’s functionality and point of interest (POI) accessibility level, the system models a user stereotype profile, which represent user’s related knowledge (which is layered with several knowledge representation structures and models shown in Fig. 1) and produce an accurate touristic recommendation plan \cite{Santos2018}. The study represent itself as a opportunity to provided needed information to people with disabilities through a hybrid tourism recommendation system. 



\section{Conclusion}

This study has explored the Big data applications and analytics and its implication  for travel accessibility. This study illustrated the importance of improving travel accessibility by recognizing the underestimated market for travel of people with disabilities. The study identify the potential of big data applications such as recommender system and big data analytics that can be used to improve travel accessibility.Both tourism practitioners and tourism researcher has recognized the
influence of big data and big data sources for tourism development. 


\begin{acks}

  The authors would like to thank Dr. Gregor von Laszewski and TAs for i523 for his
  support and suggestions to write this paper.

\end{acks}

\bibliographystyle{ACM-Reference-Format}
\bibliography{report} 

\appendix
