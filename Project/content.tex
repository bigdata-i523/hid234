\documentclass[sigconf]{acmart}

\input{format/i523}

\begin{document}


\title{Big Data Applications in the Travel Industry and its Potential in Improving Travel Accessibility}


\author{Weixuan Wang}
\affiliation{%
  \institution{Indiana University Bloomington}
  \streetaddress{School of Public Health}
  \city{Bloomington} 
  \state{Indiana} 
  \postcode{47405}
}
\email{wangweix@indiana.edu}


% The default list of authors is too long for headers}
\renewcommand{\shortauthors}{Weixuan Wang}


\begin{abstract}
Big data applications and analytics have been influencing and improving tourists' experience. Travel accessibility refers to provide access for people with disabilities or limited mobility (such as seniors), who represent a growing market in the travel industry by spending billions on leisure and business trips. This report will synthesize big data analytics and its implements in the travel industry and explore their potentials in improving travel accessibility information of the reviewer.
\end{abstract}

\keywords{i523, HID234, Big Data Analytics, Travel Accessibility, People with Disabilities}


\maketitle



\section{Introduction}
People with disabilities represented a large neglected tourism market. According to
Amadeus annual report, 15 percent of worldwide population (1 billion people) lives
with some form of disability \cite{Ama}. The World Health Organization estimates
that by 2050, 21.5 per cent of the global population
will be aged over 65. People with accessibility issues
represent a large and fast growing minority group
worldwide. According to the Open Door Organization (ODO) market report in 2015, people with disabilities spend 17.3 billion dollars annually for their own travel. Because people with disabilities usually needs a care giver or family member to accompany them when traveling, the potential economic impact could double \cite{ODO}. 

``Accessible travel] enables people with access requirements, including mobility,
vision, hearing and cognitive dimensions of access, to function independently, with
equity and dignity through the delivery of universally designed tourism products,
services and environments'' \cite{Ama}. Travel accessibility or travel for people with disabilities. Aspects that need to consider are transports, hotel, and attractions \cite{Ama}.

The ultimate aim for those involved in supporting accessible travel is to empower every individual to plan and travel independently, at their own will \cite{zhang2016}. However, the task is not a easy one. An inaccessible transport network prevents many people from going to school or studying, working, going
to the doctor, meeting friends, going shopping or to the
cinema and other activities that are taken for granted. The inaccessible transport network would also prevent people with disabilities to travel for business or leisure. Making the whole travel chain accessible, including the information and booking procedures, as well as the infrastructure and processes become a important task for travel accessibility \cite{Ama}. 



\section{Accessible Transportation and Big data}
Older adults, people with disabilities, individuals in low-income households, especially those living in rural areas can face significant mobility challenges. Concerns about getting into an accident, congestion, price of travel, access to transit, and lack of walkways are important issues for a large percentage of the population, but they tend to be more important for people with disabilities \cite{moya2016dynamic}.
For today's travel and transportation businesses, it is important to address the issue of inclusion, which is the potential to enable a broader range of people to use transportation infrastructure regardless of their individual abilities or disabilities
\cite{milo}.
Accessibility has always been a key concept in urban and regional planning for its capacity to link the activities of people and businesses to the possibilities of reaching them effectively. The concept of urban accessibility are different from travel accessibility, since the urban accessibility is focus on provide access of transportation and transit to general public, not specifically people with disabilities \cite{moya2016dynamic}. However, previous studies using big data on urban accessibility can provide some potentials for travel accessibility studies.

\section{Travel and Big data}
Most activities in the tourism industry had been generating a huge amount of data for
several years. Booking flight tickets, reserving a hotel room and renting a car all
leaves a data trail \cite{Shafiee16}. These data could add up to more than hundred of
terabytes or petabytes structured data in the conventional databases \cite{akerkar2012}.
Discussions of travel planning on online travel community, status updates and posts on
social media like Facebook and Twitter, compliments and compliant on review websites like TripAdvisor
constructs more challenging and live unstructured data that arrives at a much
faster pace than a conventional database \cite{akerkar2012}. Tourism practitioners are
trying to understand tourists' behavior by accepting and analyzing these big data 
\cite{Shafiee16}.

Airline and hotel chains have been using their big data which is the large volume of
structured information that has been produced internally \cite{MIAH2017}. Airlines and
hotels have been using this tool to analyze prices of plane ticket and hotel room
\cite{GJT14}. Moreover, airlines have optimized the details of planning for the crew and
routing \cite{Shafiee16, GJT14}. The online sector of the tourism industry has also
quickly adopted big data to improve internal decisions and understand customers 
\cite{akerkar2012}. The online sector of the industry include  meta-search engines (like 
Google), online travel agencies (like Expedia) and some information website companies that
distribute tourism information (TripAdvisor)\cite{MIAH2017}. For example, Amadeus has 
developed a program ``Amadeus Airline Cloud Availability'' that can generated special 
result and increase search for its customers and Kayak has developed a program to predict 
costs and prices for tourists\cite{Shafiee16}.

\section{Promise of big data in travel accessibility}

\subsection{Sentiment Analysis on Hotel reviews}
Sentiment analysis, which is also called opinion mining, is one of the most active
research areas in natural language processing \cite{opinion2014}. The aim of sentiment analysis is to define automatic tools able to extract subjective information from text in natural \cite{article} language, and to create structured and actionable knowledge to be used by either a decision support system or a decision maker. The sentiments of reviews, online reputation or online documents are usually categorized in positive, negative and (in some studies) neutral sentiments \cite{Garcia2012}. The main goal of the sentiment classification is to extract ``the global sentiment based on the subjectivity and the linguistic characteristics of the words within an unstructured text'' \cite{Garcia2012}. Therefore, sentiment analysis provided a framework to transform unstructured text to structured data, which make it strongly applicable to both the academic field \cite{Cam2013}. Because of the importance of sentiment analysis to business and society, it has spread from computer science to management science and the social sciences \cite{Pozzi}. As a social science field and business industry, tourism an travel studies have already been using sentiment analysis in the research.

The popularity of social media, especially review sites like TripAdvisor and blogs and wikis, leads to an enormous amount of personal reviews for travel-related information on the Web \cite{opinion2014}. More importantly, the information in these reviews is valuable to both tourists and travel and tourism practitioners for various understanding and planning processes \cite{YE20096527}. Previous studies have identified two primary approaches for sentiment analysis: methods based on the combination of lexical resources and Natural Language Processing (NLP) techniques; and machine learning approaches \cite{Garcia2012}. Since 2009, researchers have been using machine learning methods in the natural language processing (support vector machine (SVM), Naïve Bayes, and the N-gram model) to do sentiment analysis on TripAdvisor reviews \cite{YE20096527}. 
Their study analyzed online reviews related to travel destinations, using different supervised machine learning algorithms The
algorithms to evaluate the reviews about seven popular travel destinations in Europe and North America \cite{YE20096527}.

Thee etBlogAnalysis project developed a combined crawler /sentiment extraction application for the tourism industry, which used a simple and robust linguistic parsing methodology with information and terminology extraction methods in order to determine relevant utterances on expression level \cite{opinion2014}. It will also provide a warning for tourism operator such as a hotel, if too many negative entries have been generated by their reviewers \cite{Garcia2012}.

In tourism studies, sentiment analysis has been compared to traditional qualitative analytic methods. A previous study compared three alternative approaches for mining consumer sentiment (manual content coding, corpus-based semantic analysis, and stance-shift analysis) from large amounts of qualitative data found in online travel reviews \cite{Farm}. They applied three different approaches to study consumers' reaction to farm stays in order to demonstrate how large volumes of qualitative data can be analyzed quantitatively in a relatively efficient and reliable way \cite{Farm}.  Manual content coding is the same as traditional the content analysis approach involving two researchers collaborated in a manual coding process designed to extract consumer likes and dislikes from the qualitative data \cite{Farm}. According to the comparison, computer generated sentiment analysis such as stance-shift analysis processing on both syntax and lexicon assures the coding maintains the statement's context identifying what is important to the informants by the way they express their comments. Most importantly, stance-shift analysis does not categorize what the researcher thinks is important in reviewer's words \cite{Farm}. The study suggested by combining different approaches in sentiment analysis such as using stance-shift analysis first identifies the significant word segments then using corpus-based semantic analysis detects key themes in those segments helps uncover narrative themes of consumer experiences in large qualitative databases \cite{Farm}.

Sentiment analysis will help researchers to better understand people's travel experience, however, there are few studies have been done to identify demographic information of the reviewer and compare the sentiment analysis result across different demographic. A recent invention present the possibility of identifying demographic characteristics while conducting sentiment analysis. The invention consist of a product or service review to determine demographic information of the reviewer \cite{Bhatt2014}. A sentiment text analysis is performed on the product or service review, wherein the sentiment text analysis examines the product or service review to determine a sentiment of the product or service review. The sentiment of the product or service review is categorized based on the demographic information of the reviewer \cite{Bhatt2014}. This invention present the promise of using sentiment analysis on the travel experience of people with disabilities. However, challenges still remain for research of UGC generated by people with disabilities, such as the challenge presented by privacy concerns of personal data online \cite{lazar}. 


\subsection{Recommender System}
Information Communication Technologies (ICTs) have been transforming tourism business
globally and revolutionizing the world of Tourism. It transforms tourism from a
labor-intensive to an information-intensive industry \cite{Williams201787}.
Tourists influence by the developments in search engines, network speed and capacity 
have been using use technologies for better planning and experiencing their
trips \cite{XIE2017101}. In addition, ICTs enable travelers to access reliable and
accurate information and make reservations faster, cheaper and more convenient than
the traditional way \cite{chung2009}.

The development of ICTs also enables Internet users to both create and distribute
information (especially multimedia information), which is called user-generated 
content (UGC) or consumer-generated content (CGC) \cite{chung2009}.

Nowadays tourists faces a very challenging task of trip preparation because of the
huge amount of information available on the Web about tourism and leisure activities.
Recommender systems becomes essential for tourists and tourism operates. For tourists,
recommender systems can be a useful tools to help them make decision for travel
planning, such as the choices of destinations, attractions, accommodations and
restaurants. As for tourism operators, it can be a great marketing opportunities for
them to reach a variety of targeted potential consumers. Complex problems such as
automated planning, semantic knowledge management, group recommendation or
context-awareness have by now been heavily studied in this area \cite{morenorecommender}. 

There are already several tourism recommender system available for general public. TIP and Heracles systems provide recommendation service through mobile devices for tourism, through implement hybrid algorithms to calculate tourist preferences, using the defined tourist profile and location data \cite{morenorecommender}. Crumpet provides new information delivery services for a variety of different tourist population based on location aware services, personalized user interaction, accessible multi-media mobile communication and smart component-based middleware that uses Multi-Agent Technology \cite{Santos2018}. CATIS is a Web based tourist information system using context-awareness, which include context elements such as location, time of day, speed, direction of travel and personal preferences. This system provided information to tourists relevant to his or her location and time \cite{Santos2018}. TravelWithFriends using group recommendation service, the first step is to build a recommendation list for each user and to merge them to obtain a destinations shortlist. Afterwards, each group member rates all these options and a Borda count is used to determine the best five destinations to be recommended \cite{morenorecommender}.

Classical recommender systems filter the domain items according to a particular user, using his or her demographic data, past ratings or purchasing history \cite{LU201512}. This approach are used to recommend specific items such as books, songs or films \cite{LU201512}. However, it may not be suitable for travel activities, since most of time travel is an activity that involves a group of people (such as family members, friends). Therefore, it is necessary to take into account the different preferences of all members of travel group when providing recommendations \cite{morenorecommender}. Previous studies and technology reports have identified two primary options for group recommendation: the first one is to merge the lists of items recommended to each group member, or creating a group profile with everyone’s preferences and then compute a single list of group recommendations \cite{Garcia2009}. The second option’s first step is the same as the previous option, by constructing of a list of recommendations for each group member. In a second step though, an automatic consensus-reaching process is applied, in which individual preferences are continuously updated until a high degree of agreement between all the group members is reached \cite{Garcia2009}.

The use of semantic domain knowledge in the recommendation process has heavily increased in recent years. Previous studies have defined the semantic similarity between two concepts is as ``the ratio between the number of different ancestors and the total number of ancestors of both concepts’’ \cite{morenorecommender}. The items to be recommended are clustered according to this semantic similarity and the recommendation procedure selects the best item from random clusters \cite{Santos2018}. Previous study has shown that this procedure keeps the accuracy and increases the diversity of the results \cite{morenorecommender}. Semantic information can also be used to determine the items to be recommended in a personalized visit to a museum or destinations, by using a shortest-path semantic distance to determine which museum objects or attractions should be recommended to the user \cite{morenorecommender}. 

Previous study also proposed a hybrid tourism recommendation system for persons suffering from physical or intellectual limitation. This proposed recommendation system is not simply trying to improve experience, but to create in users the confidence that despite of their limitations they can visit and experience certain places without being afraid, and to help them to live a touristic experience. By identifying the user’s functionality and point of interest (POI) accessibility level, the system models a user stereotype profile, which represent user’s related knowledge (which is layered with several knowledge representation structures and models shown in Fig. 1) and produce an accurate touristic recommendation plan \cite{Santos2018}. The study represent itself as a opportunity to provided needed information to people with disabilities through a hybrid tourism recommendation system. 



\section{Conclusion}

This study has explored the Big data applications and analytics and its implication  for travel accessibility. This study illustrated the importance of improving travel accessibility by recognizing the underestimated market for travel of people with disabilities. The study identify the potential of big data applications such as recommender system and big data analytics that can be used to improve travel accessibility.Both tourism practitioners and tourism researcher has recognized the
influence of big data and big data sources for tourism development. 


\begin{acks}

  The authors would like to thank Dr. Gregor von Laszewski and TAs for i523 for their
  supports and suggestions to write this paper.

\end{acks}

\bibliographystyle{ACM-Reference-Format}
\bibliography{report} 

\appendix


\end{document}
